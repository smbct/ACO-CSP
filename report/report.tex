\documentclass{article}

\usepackage[utf8]{inputenc}
\usepackage[T1]{fontenc}
\usepackage[frenchb]{babel}

\usepackage{a4wide}

\usepackage{amsmath}
\usepackage{amssymb}
\usepackage{amsthm}

\usepackage{hyperref}


\title{Swarm intelligence: Ant colony optimization project}
\author{Samuel Buchet: 000447808}
\date{June 2017}

\begin{document}

\maketitle

\section{Introduction}

\section{Solution}

A solution is represented as a string.
In order to be more efficient for the computation, a matrix used.
The matrix contains one list for each string of the set.
The element at position i in the list is a number equal to the hamming distance
on the sub lists of size $i+1$.
The final distance can be read on the lste element of the list.
A variable is also used to keep the index of the last distance.

\section{Greedy algorithm}

The aco depends on a greedy parameter for each choice of the solution.
The problem aims to minimize the maximum distance.
A greedy criteria could consist in choosing the character that minimizes the distance.
For each position, we can assign a score to each character to the alphabet.
This score would consist in counting how many times the character appears in the strins for a given position.
We can also take into account the number of characters in common from the begining of the string. \newline

It looks like the greedy solution is very good (better than what the algorithm is able to find).
There must be a problem because it is actually good when we take the min. \newline

Actually, there was a problem.
The greedy score was inverted so that it tried to obtain the worst possible score.
And It worked, the greedy solution was worst than the random one.
So the next problem is that the greedy information cannot be used properly.
The greedy scores are very close to each other, the roulette does not really have a good effect.
If we scale the scores, the greedy one is note often chosen.
Idea: use the exponential function.

\end{document}

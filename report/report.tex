\documentclass{article}

\usepackage[utf8]{inputenc}
\usepackage[T1]{fontenc}

\usepackage{a4wide}

\usepackage{amsmath}
\usepackage{amssymb}
\usepackage{amsthm}

\usepackage{hyperref}

\title{Swarm intelligence: Ant colony optimization project}
\author{Samuel Buchet: 000447808}
\date{June 2017}

\begin{document}

\maketitle

\section{Introduction}

In this project, two anto colony optimization algorithms are implmented to solve the Closest String Problem.
The two algorithms implemented are the Max Min Ant System algorithm (MMAS) and the Ant Colony System algorithm (ACS).
In the following report, the two algorithms are first described.
After that, the two algorithms are compared.
Finally, a local search is used to improve the results.

\section{Implementation of the algorithms}

In this section, the Max Min Ant System and the Ant Colony System algorithms are described.
These two agorithms also use a heuristic information which is explained below.

\subsection{Heuristic information}

In \cite{aco_csp}, no heuristic information is used.
However, most of the ant colony algorithms use this kind of information.
In addition, it can be easily computed for a lot of problems.
It has been decided to define one for the closest string problem. \newline

The goal of the closest string problem is to minimize the maximum hamming distance between the solution and the set of strings.
To minimize the distance between the solution and the set of strings, a greedy decision consists in adding into the solution the character which appears the most in the set of strings at a given position.
Indeed, by using this character, the distance might be equal to 0 at this position.
For each character at each position of the string, a greedy score can be defined as the frequency of the character.\newline

However, the frequencies might be very close to each other.
As a result, if this score is used in the probabilities, the probabilities of getting the best greedy character would be very low.
To solve this problem, the exponential function is applied to the score after scaling it.
The final score of character $j$ at position $i$ is equal to: $score_{ij} = \frac{exp(5*frequency_{ij})*1.5}{max_j(frequency_{ij})}$ where $frequency_{ij}$ is the number of occurences of the character $j$ at position $i$ in the set of string of the problems.
The parameters have been chosen after testing different values.

\subsection{Basic ant system algorithm}

Both variants of the ant colony algorithms implemented in this project rely on the same basis.
At each iteration, the general algorithm builds $n$ solutions (with $n$ equal to the size of the population) and then updates the pheromones using the formula: $\tau_{ij}(t) = (1-\rho)*\tau(t-1) + \sum \limits_{k=1}^m \Delta \tau_{ij}^k$  where $\tau_{ij}(t)$ is the amount of pheromone at position $i$ for character $j$ after t iterations, $\rho$ is the evaporation rate and $\tau_{ij}^k$ is the quantity of pheromone deposited by ant $k$ if this ant have chosen character $j$ at position $i$. \newline

As seen in \cite{aco_csp}, the amount of pheromone deposited by the ants is equal to $1-\frac{HD}{m}$ where HD is the maximum hamming distance of the ant and m is the number of strings of the problem.
This quantity has been used in the two implementations. \newline

To build a solution, each character is chosen according to a probabilty.
The probability of choosing the character $j$ at position $i$ is equal to $ \frac{greedyScore_{ij}^{\alpha} * \tau_{ij}^{\beta}}{\sum probas}$ where $greedyScore$ is the greedy score described previously and $\tau_{ij}$ is the amount of pheromone at position $i$ for the character $j$. 


\subsection{Max Min Ant System}



\subsection{Ant Colony System}


\section{pheromones and probas}

To obtain a completely greedy solution, the alpha should be very big and negative.
With this settings, the pheromones should be greater or equal to one, otherwise the result is too big.

\section{Parameter tuning}

MaxMin gives good results, but still to improve.
main problem: resetting pheromone has no effect maybe because the max limit is too low.
Something weird with the pheromones in maxmin : if it does not deposit pheromones, it still improve well.\newline \newline

ACS: Initially, high initial pheromones  bad result
Then try with the value 1, much better results, pheromones really used, very close to the best solution.
Always improving

\section{Local search}

if the value is not close to the optimal, the local search improves a lot.
Otherwise, it improves just a little.

\begin{thebibliography}{2}

\bibitem{aco_csp} Simone Faro and Elisa Pappalardo.
Ant-CSP: An Ant Colony Optimization Algorithm for the Closest String Problem, pages 370-381.
Springer Berlin
Heidelberg, Berlin, Heidelberg, 2010.

\end{thebibliography}

\end{document}
